%%%%%%%%%%%%%%%%%%%%%%%%%%%%%%%%%%%%%%%%%
% Freeman Curriculum Vitae
% XeLaTeX Template
% Version 2.0 (19/3/2018)
%
% This template originates from:
% http://www.LaTeXTemplates.com
%
% Authors:
% Vel (vel@LaTeXTemplates.com)
% Alessandro Plasmati
%
% License:
% CC BY-NC-SA 3.0 (http://creativecommons.org/licenses/by-nc-sa/3.0/)
%
% !TEX program = xelatex
% NOTICE: This template must be compiled with XeLaTeX, the line above should
% ensure this happens automatically but if it doesn't you will need to specify 
% XeLaTeX as the engine in your editor or script
% 
%%%%%%%%%%%%%%%%%%%%%%%%%%%%%%%%%%%%%%%%%

% ----------------------------------------------------------------------------------------
%	PACKAGES AND OTHER DOCUMENT CONFIGURATIONS
% ----------------------------------------------------------------------------------------

\documentclass[10pt]{article} % Font size, can be: 10pt, 11pt or 12pt

\input{structure.tex} % Include the file that specifies the document structure

% Headers and footers can be added with the \lhead{} \rhead{} \lfoot{} \rfoot{} commands
% Example right footer:
% \rfoot{\color{headings}{\sffamily Last update: \today. Typeset with Xe\LaTeX}}

% ----------------------------------------------------------------------------------------

\begin{document}

\begin{paracol}{2} % Begin the multi-column environment

  % ----------------------------------------------------------------------------------------
  %	NAME AND CURRICULUM VITAE TEXT
  % ----------------------------------------------------------------------------------------

  \parbox[top][0.12\textheight][c]{\linewidth}{ % Parbox to hold the author name and CV text; fixed height to match the coloured box to the right, centred vertically and full line width
    \vspace{-0.04\textheight} % Reduce whitespace above the parbox to separate it from the main content
    \centering % Centre text
    {\sffamily\Huge Loïc Lemaître}\\\medskip % Your name
    {\Huge\color{headings}\cvtextfont Développeur Web}
  }

  % ----------------------------------------------------------------------------------------
  %	WORK EXPERIENCE
  % ----------------------------------------------------------------------------------------

  \section{Expérience professionnelle}

  % Blank \workposition command to add another job:

  % \workposition{} % Duration
  % {} % FT/PT (full time or part time)
  % {} % Employer
  % {} % Job title
  % {} % Description

  % All 5 parameters must be supplied but any can be empty if you don't need them

  % ------------------------------------------------

  \workposition{2013 - auj.} % Duration
  {CDI} % FT/PT (full time or part time)
  {Doqboard {\small(\textit{https://doqboard.com})} {\small- Rouen (télétravail depuis 2015)}} % Employer
  {Développeur} % Job title
  { Création d'une application de gestion de projets en chirurgie : formulaire sur mesure de collecte de données, statistiques, export. Web {\small(\textit{depuis 2018})} : React (Javascript) / Django (Python) / PostgreSQL. Bureau multi-plateforme {\small(\textit{jusqu'en 2018})} : wxWidgets (C++) / JHipster (Java) / PostgreSQL.
  } % Description

  % ------------------------------------------------

  \workposition{2012 -- 2013} % Duration
  {Contrat pro} % FT/PT (full time or part time)
  {Atelier Bois Construction {\small- Verneuil sur Avre}} % Employer
  {Apprenti charpentier} % Job title
  {Charpente traditionnelle. Ossature bois. Bardage intérieur et extérieur. Terrasses.}  % Description

  % ------------------------------------------------

  \workposition{2010 -- 2012} % Duration
  {CDI} % FT/PT (full time or part time)
  {Jeulin {\small- Evreux}} % Employer
  {Développeur} % Job title
  {Création d'un langage informatique compilé / interprété pour un automate programmable embarqué: C. Logiciels pédagogiques: wxWidgets (C++)} % Description

  \workposition{2008 -- 2010} % Duration
  {CDI} % FT/PT (full time or part time)
  {INRIA {\small- Villeurbanne}} % Employer
  {Développeur} % Job title
  {Logiciels de simulation open source : C, embarqué, Linux. WSim : émulateur de plateformes programmables sans fils {\small(\textit{http://wsim.gforge.inria.fr})}. WSNet : simulateur de réseaux de plateformes sans fils {\small(\textit{http://wsnet.gforge.inria.fr)})}} % Description

  \workposition{2007 -- 2008} % Duration
  {CDI} % FT/PT (full time or part time)
  {Devoteam {\small- Lyon}} % Employer
  {Développeur} % Job title
  {Paramétrage avancé de logiciels de gestion pour entreprises (HP Service Center).} % Description

  \workposition{2006 -- 2007} % Duration
  {CDI} % FT/PT (full time or part time)
  {Devoteam Consulting {\small- Levallois-Perret}} % Employer
  {Consultant télécom} % Job title
  {Conseil pour les entreprises (téléphonie sur IP et réseaux).} % Description

  % ------------------------------------------------

  \vspace{-\baselineskip}\medskip % Standardise the whitespace after this section and before the next (the custom command adds too much otherwise)

  \switchcolumn % Switch to the next paracol column

  % ----------------------------------------------------------------------------------------
  %	COLOURED CONTACT DETAILS BOX
  % ----------------------------------------------------------------------------------------

  \parbox[top][0.12\textheight][c]{\linewidth}{ % Parbox to hold the colour box; fixed height to match the name/CV text to the left, centred vertically and full line width
    \vspace{-0.04\textheight} % Reduce whitespace above the parbox to separate it from the main content
    \colorbox{shade}{ % Create the coloured box
      \begin{supertabular}{p{0.05\linewidth}|p{0.775\linewidth}} % Start a table with two columns, the table will ensure everything is aligned
        \raisebox{-1pt}{\faHome} & Charlat Haut - 19 500 Collonges-la-Rouge \\ % Address
        \raisebox{-1pt}{\faPhone} & 06 62 57 34 88 \\ % Phone number
        \raisebox{0pt}{\small\faEnvelope} & \href{mailto:loic.lemaitre@gmail.com}{loic.lemaitre@gmail.com} \\ % Email address
        \raisebox{-1pt}{\small\faGitlab} & \href{https://gitlab.com/Moki19}{https://gitlab.com/Moki19} \\ % Gitlab
        \raisebox{-1pt}{\small\faGithub} & \href{https://github.com/Moki-19}{https://github.com/Moki-19} \\ % Github
        % \raisebox{-1pt}{\faGithub} & \href{https://github.com/username}{https://github.com/username} \\ % GitHub profile
        % \raisebox{-1pt}{\faLinkedinSquare} & \href{https://www.linkedin.com/in/username}{https://www.linkedin.com/in/username} \\ % LinkedIn profile
        % See fontawesome.pdf in the fonts folder for all icons you can use
      \end{supertabular}
    }
  }

  % ----------------------------------------------------------------------------------------
  %	EDUCATION
  % ----------------------------------------------------------------------------------------

  \section{Formation}

  % Blank \educationentry{} command to add another degree:

  % \educationentry{} % Duration
  % {} % Degree
  % {} % Honours, achievements or distinctions (e.g. first class honours)
  % {} % Department
  % {} % Institution

  % All 5 parameters must be supplied but any can be empty if you don't need them

  % ------------------------------------------------

  \begin{supertabular}{rl} % Start a table with two columns, the table will ensure everything is aligned

    % ------------------------------------------------

    \educationentry{2012 -- 2013} % Duration
    {CAP Charpente} % Degree
    {} % Honours, achievements or distinctions (e.g. first class honours)
    {} % Department
    {Compagnins du devoir - Rouen} % Institution

    % ------------------------------------------------

    \educationentry{2000 - 2006} % Duration
    {Ingénieur Génie Mathématique} % Degree
    {} % Honours, achievements or distinctions (e.g. first class honours)
    {} % Department
    {INSA - Rouen} % Institution

    % ------------------------------------------------

    \educationentry{2000} % Duration
    {Bac Scientifique} % Degree
    {} % Honours, achievements or distinctions (e.g. first class honours)
    {} % Department
    {Spécialité mathématiques} % Institution

    % ------------------------------------------------

  \end{supertabular}

  % ----------------------------------------------------------------------------------------
  %	COMPUTER SKILLS
  % ----------------------------------------------------------------------------------------

  \section{Compétences}

  % Example \tableentry{} command to add another line:

  % \tableentry{Heading}{Content}{spaceafter}

  % All 3 parameters must be supplied but any can be empty if you don't need them
  % A "spaceafter" value in the third parameter will add some vertical space -- this is to be used between headings

  % ------------------------------------------------

  \begin{supertabular}{rl} % Start a table with two columns, the table will ensure everything is aligned

    % ------------------------------------------------

    \tableentry{Programmation}{Javascript, Python, C, C++, PHP,}{}
    \tableentry{}{HTML, CSS,  Bash, Batch, Elisp}{spaceafter}

    % ------------------------------------------------

    \tableentry{Frameworks}{React, Django RF, wxWidgets}{spaceafter}

    % ------------------------------------------------

    \tableentry{Bibliothèques}{Material-UI, LibCurl, LibXml,}{}
    \tableentry{}{Eslint, Jest}{spaceafter}

    % ------------------------------------------------

    \tableentry{Base de données}{PostgreSQL, MySQL}{spaceafter}

    % ------------------------------------------------

    \tableentry{IDE}{Emacs}{spaceafter}

    % ------------------------------------------------
    \tableentry{OS}{Linux, MAC OS, Windows}{spaceafter}

    % ------------------------------------------------

    \tableentry{Embarqué}{MCU Microship PIC/TI MSP430,}{}
    \tableentry{}{Radios TI CC2420/1100}{spaceafter}

    % ------------------------------------------------

  \end{supertabular}

  % ----------------------------------------------------------------------------------------
  %	COMMUNICATION SKILLS
  % ----------------------------------------------------------------------------------------

  \section{Activités personnelles}

  % Example \tableentry{} command to add another line:

  % \tableentry{Heading}{Content}{spaceafter}

  % All 3 parameters must be supplied but any can be empty if you don't need them
  % A "spaceafter" value in the third parameter will add some vertical space -- this is to be used between headings

  % ------------------------------------------------

  \begin{supertabular}{rl} % Start a table with two columns, the table will ensure everything is aligned

    % ------------------------------------------------

    \tableentry{Sports}{VTT (club), tennis, ski}{}

    % ------------------------------------------------

    \tableentry{Autres}{Jardinage, musique, randonnée}{}

    % ------------------------------------------------

  \end{supertabular}

  \medskip % Extra whitespace before the next section

  % ----------------------------------------------------------------------------------------

\end{paracol}

% ----------------------------------------------------------------------------------------

\end{document}

%%% Local Variables:
%%% coding: utf-8
%%% mode: latex
%%% TeX-engine: xetex
%%% End: